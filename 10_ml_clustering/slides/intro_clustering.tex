\documentclass[12pt,aspectratio=169]{beamer}

\usetheme{metropolis}

\definecolor{mDarkBrown}{HTML}{FF5722}
\definecolor{mDarkTeal}{HTML}{263238}
\definecolor{mLightBrown}{HTML}{FF5722}

\usepackage{booktabs}
\usepackage{graphicx}
\usepackage{hyphenat}
\usepackage{multirow}
\usepackage{nicefrac}
\usepackage[normalem]{ulem}

\usepackage{pifont}
\newcommand{\cmark}{\ding{51}}
\newcommand{\xmark}{\ding{55}}

\usepackage{minted}
\usemintedstyle{tango}
\newminted[bash]{bash}{%
    autogobble,
    bgcolor=mDarkTeal!10,
    linenos
}
\newminted[py3]{python}{%
    python3,
    autogobble,
    bgcolor=mDarkTeal!10,
    linenos
}
\newminted[sql]{sql}{%
    autogobble,
    bgcolor=mDarkTeal!10,
    linenos
}

\usepackage{polyglossia}
\setdefaultlanguage[variant=british]{english}
\usepackage[english=british]{csquotes}

\defaultfontfeatures{Ligatures=TeX}
\setmainfont{Lucida Sans OT}
\setsansfont[Scale=MatchLowercase]{Lucida Sans OT}
\setmonofont[Scale=MatchLowercase]{Lucida Console DK}

\usepackage{mathspec}
\setmathsfont(Digits,Latin,Greek)[Numbers={Lining,Proportional}]{Lucida Bright Math OT}

\newcommand{\mat}[1]{\ensuremath{\mathbf{#1}}}

\newcommand{\R}{\ensuremath{\mathbb{R}}}

\newcommand{\E}[1]{\ensuremath{\mathbb{E}\!\left[ #1 \right]}}
\newcommand{\V}[1]{\ensuremath{\mathbb{V}\!\left[ #1 \right]}}
\newcommand{\Prob}[1]{\ensuremath{\Pr\!\left( #1 \right)}}
\newcommand{\Normal}[2]{\ensuremath{\mathcal{N}\!\left( #1, #2 \right)}}
\newcommand{\simiid}{\ensuremath{\overset{\text{\tiny i.i.d.}}{\sim}}}

\DeclareMathOperator{\logit}{logit}

\author{Gianluca Campanella}
\date{}



\title{Introduction to clustering}

\begin{document}

\maketitle

\begin{frame}{Contents}
    \tableofcontents[hideallsubsections]
\end{frame}

\section{Clustering}

\begin{frame}{Classification versus clustering}
    \only<1>{%
        \begin{block}{Classification}
            \begin{itemize}
                \item Data are `labelled' $\rightarrow$ \alert{supervised}
                \item[$\rightarrow$] Find a `rule' that assigns labels to new
                                     observations
            \end{itemize}
        \end{block}
        \vfill
        \begin{block}{Clustering}
            \begin{itemize}
                \item Data are `unlabelled' $\rightarrow$ \alert{unsupervised}
                \item[$\rightarrow$] Identify structure and patterns
            \end{itemize}
        \end{block}}
    \only<2>{%
        \begin{block}{Idea}
            \begin{itemize}
                \item Group observations that are `close'
                      (high intra\hyp{}cluster similarity)
                \item Identify `natural' groupings
                      (low inter\hyp{}cluster similarity)
            \end{itemize}
        \end{block}
        \vfill
        \begin{block}{Types of clustering}
            \begin{itemize}
                \item \textbf{Hard}: each observation belongs to
                      \alert{exactly one} cluster
                \item \textbf{Soft} (or \textbf{fuzzy}): observations may belong
                      to multiple clusters
                \item \textbf{Hierarchical}: observations belong to `concentric'
                      clusters
            \end{itemize}
        \end{block}}
\end{frame}

\section{$k$-means}

\begin{frame}{$k$-means}
    \only<1>{%
        Given the number of clusters $k$\ldots
        \begin{itemize}
            \item Select $k$ centroids (e.g.\ $k$ observations at random) \\[\medskipamount]
            \item For each observation:
                  \begin{itemize}
                      \item Determine distances to the centroids
                      \item Reassign to the closest centroid \\[\medskipamount]
                  \end{itemize}
            \item Recompute the centroids \\[\medskipamount]
            \item Repeat until no observations move group
        \end{itemize}}
    \only<2>{%
        \begin{block}{Questions}
            \begin{itemize}
                \item How do we define similarity?
                \item How many clusters do we use?
            \end{itemize}
        \end{block}}
\end{frame}

\begin{frame}{Curse of dimensionality}
    As the number of variables (coordinates) increases\ldots
    \begin{itemize}
        \item The volume of the space increases
        \item Pairwise distances become more similar $\rightarrow$ sparsity
        \item Some samples have huge neighbourhoods $\rightarrow$ `hubs'
    \end{itemize}
\end{frame}

\end{document}

